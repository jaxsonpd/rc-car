\chapter{Test programs}
\label{test-programs}

There are a number of test programs in the directory
\wfile{test-apps}.  Where possible these are written to be
independent of the target board using configuration files (see
\protect\hyperref[configuration]{configuration}).


\section{USB}
\label{usb-interfacing}

To help debug your programs, it is useful to be able to print values.
This can be achieved by redirecting stdio using the \wikiref{USB
  CDC}{USB CDC} protocol.  For example, here's an example program
\wfile{test-apps/usb_serial_test1/usb_serial_test1.c}.

\inputminted{C}{../../src/test-apps/usb_serial_test1/usb_serial_test1.c}

To get this program to work you need to compile it and program the
SAM4S using:
%
\begin{minted}{bash}
$ cd wacky-racers/src/test-apps/usb_serial_test1
$ make program
\end{minted}

You then need to connect your computer to the USB connector on your PCB.
If you are running Linux, run:
%
\begin{minted}{bash}
$ dmesg
\end{minted}

This should say something like:
%
\begin{verbatim}
Apr 30 11:03:50 thing4 kernel: [52704.481352] usb 2-3.3: New USB device found, idVendor=03eb, idProduct=6202
Apr 30 11:03:50 thing4 kernel: [52704.481357] usb 2-3.3: New USB device strings: Mfr=1, Product=2, SerialNumber=3
Apr 30 11:03:50 thing4 kernel: [52704.482060] cdc_acm 2-3.3:1.0: ttyACM0: USB ACM device
\end{verbatim}

Congrats if you see \texttt{ttyACM0:\ USB\ ACM\ device}!  If not, see
\wikiref{USB_debugging}{USB debugging}.

You can now run a \wikiref{Serial_terminal_applications}{serial
  terminal program}. For example, on Linux:
%
\begin{minted}{bash}
$ gtkterm -p /dev/ttyACM0
\end{minted}

All going well, this will repeatedly print 'Hello world'.

If run Linux and get an error 'device is busy', it is likely that the
ModemManager program has automatically connected to your device on the
sly. This program should be disabled on the ECE computers. For more
about this and using other operating systems, see \wikiref{USB
CDC}{USB CDC}.

\section{PWM}
\label{pwm-test}

The program \wfile{test-apps/pwm_test2/pwm_test2.c} provides an
example of driving PWM signals.

\inputminted{C}{../../src/test-apps/pwm_test2/pwm_test2.c}

Notes:
%
\begin{enumerate}
\item
  This is for a different H-bridge module that requires two PWM signals
  and forward/reverse signals. You will need to generate four PWM
  signals or be clever with two PWM signals.
\item
  The \code{pwm_cfg_t} structure configures the frequency, duty
  cycle and alignment of the output PWM.
\item The frequency is likely to be too high for your motor.
\item \code{pwm_channels_start} is used to start the PWM channels simultaneously.  
\end{enumerate}

The most likely problem is that you have not used a PIO pin that can
be driven as a PWM output. The SAM4S can generate four independent
hardware PWM signals. See \wfile{mat91lib/pwm/pwm.c} for a list of
supported PIO pins. Note, \pin{PA16}, \pin{PA30}, and \pin{PB13} are
different options for PWM2.

To drive the motors you will need to use a bench power supply. Start
with the current limit set at 100\,mA maximum in case there are any
board shorts.  When all is well, you can increase the current limit;
you will need at least 1\,A.

\section{IMU}
\label{imu-test}

The MPU9250 IMU connects to the SAM4S using the I2C bus (aka TWI bus).
The program \wfile{test-apps/imu_test1/imu_test1.c} provides an
example of using the MPU9250 IMU.  All going well, this prints three
16-bit acceleration values per line to USB CDC. Tip your board over,
and the the third (z-axis) value should go negative since this
measures the effect of gravity on a little mass inside the IMU pulling
on a spring.

If you get `ERROR: can't find MPU9250!', the main reasons are:

\begin{enumerate}
\item
  You have specified the incorrect address.  Use 0x68 for
  \code{MPU_ADDRESS} in \file{target.h} if the AD0 pin is connected to
  ground otherwise use 0x69.
\item
  You are using TWI1. The \pin{PB4} and \pin{PB5} pins used by TWI1
  default to JTAG pins. See
  \protect\hyperref[disabling-jtag-pins]{disabling JTAG pins}.
\end{enumerate}
%
See also \protect\hyperref[checking-IMU]{IMU checking}.

Other problems:
%
\begin{enumerate}
\item If you reset the SAM4S in the middle of a transaction with the
IMU, the IMU gets confused and holds the \pin{TWD/SDA} line low. This
requires recycling of the power or sending out some dummy clocks on
the \pin{TWCK/SCL} signal.
\end{enumerate}


\section{Radio}
\label{radio-test}

The program \wfile{test-apps/radio_tx_test1/radio_tx_test1.c} provides
an example of using the radio as a transmitter.

The companion program
\wfile{test-apps/radio_rx_test1/radio_rx_test1.c} provides an example
of using the radio as a receiver.

Notes:
%
\begin{enumerate}
\item
  Both programs must use the same RF channel and the same address. Some
  RF channels are better than others since some overlap with WiFi
  and Bluetooth. The address is used to distinguish devices
  operating on the same channel. Note, the transmitter expects an
  acknowledge from a receiver on the same address and channel.
\item
  The radio `write` method blocks waiting for an auto-acknowledgement
  from the receiver device. This acknowledgement is performed in
  hardware. If no acknowledgement is received, it retries for up to 15
  times. The auto-acknowledgement and number of retries can be
  configured in software.

\item If the program hangs in the \code{panic} loop, there is no
  response from the radio module, check SPI connections and see
  \hyperref[debugging_Spi]{SPI debugging}.
  
\end{enumerate}

If you cannot communicate between your hat and racer boards, try
communicating with the radio test modules Scott Lloyd has in the SMT
lab.


\section{ADC}
\label{ADC}

\wfile{test-apps/adc_usb_serial_test2/adc_usb_serial_test2.c} shows
how to read from two multiplexed ADC channels.  For more details
see \wfile{mat91lib/adc/adc.h}.

\inputminted{C}{../../src/test-apps/adc_usb_serial_test2/adc_usb_serial_test2.c}

The ADC can be also set up to stream data continuously but you need to
use interrrupts or DMA.


\section{Pushbutton}
\label{pushbutton}

\wfile{test-apps/button_test1/button_test1.c} shows the use of a
simple button driver to read a pushbutton.  This driver does not debounce the button.

\inputminted{C}{../../src/test-apps/button_test1/button_test1.c}

\wfile{test-apps/button_test2/button_test2.c} shows the use of a
simple button driver to read a pushbutton.  This driver does button
debouncing and state-transition detection.  For more details see
\wfile{mmculib/button/button.h}.

\inputminted{C}{../../src/test-apps/button_test2/button_test2.c}


\section{LEDtape}
\label{ledtape}

\wfile{test-apps/ledtape_test1/ledtape_test1.c} shows how to drive the
LEDtape.

\inputminted{C}{../../src/test-apps/ledtape_test1/ledtape_test1.c}


\wfile{test-apps/ledtape_test2/ledtape_test2.c} uses a more
sophisticated means of controlling the LEDs.

\inputminted{C}{../../src/test-apps/ledtape_test2/ledtape_test2.c}

