\chapter{Git}
\label{git}

To properly use git you should commit and push often. The smaller the
changes and the more often you make per commit, the smaller the chance
of the dreaded merge conflict.


\section{Typical workflow}

\begin{enumerate}
\item Edit file

\item Save changes  

\item Check differences
  
\begin{minted}{bash}  
    $ git diff .
\end{minted}
  
\item Commit changes 

\begin{minted}{bash}  
    $ git commit -m ``Commit message''
\end{minted}

Note, you should commit about every 15 min, preferably when you have
made a single functional change.  Ideally each commit should be
self-contained.
  
\item Push changes to server

\begin{minted}{bash}    
    $ git push
\end{minted}

The more often you push, the lower the chance that you will get a
merge conflict.
  
\end{enumerate}


\section{Diff, status, blame, log}

The diff command is useful to determine what changes you made.

The status command says which files have been modified and what you
should do, say when you get a merge conflict.

The blame command is useful to determin who authored each line of code.

The log command shows all the previous commit messages.


\section{Pulling from upstream}
\label{git-pulling-from-upstream}

To be able to get updates if the template project is modified you will
need to:

\begin{minted}{bash}
$ cd wacky-racers 
$ git remote add upstream https://eng-git.canterbury.ac.nz/wacky-racers/wacky-racers-2021.git  
\end{minted}

Again if you do not want to manually enter your password (and have
ssh-keys uploaded) you can use:
%
\begin{minted}[breaklines]{bash}
$ cd wacky-racers 
$ git remote add upstream git@eng-git.canterbury.ac.nz:wacky-racers/wacky-racers-2021.git
\end{minted}

Once you have defined the upstream source, to get the updates from the
main repository use:
%
\begin{minted}{bash}
$ git pull upstream master
\end{minted}

If you enter the wrong URL make a mistake, you can list the remote
servers and delete the dodgy entry using:

\begin{minted}{bash}
$ git remote -v
$ git remote rm upstream
\end{minted}

Note, \textbf{origin} refers to your group project and \textbf{upstream}
refers to the template project that origin was forked from.

\section{Merging}
\label{git-merging}

The bane of all version control programs is dealing with a merge
conflict. You can reduce the chance of this happening by committing and
pushing faster than other people in your group.

If you get a message such as:

\begin{verbatim}
From https://eng-git.canterbury.ac.nz/wacky-racers/wacky-racers-2021
 * branch            master     -> FETCH_HEAD
error: Your local changes to the following files would be overwritten by merge:
    src/test-apps/imu_test1/imu_test1.c
Please, commit your changes or stash them before you can merge.
\end{verbatim}

what you should so is:

\begin{minted}{bash}
$ git stash
$ git pull
$ git stash pop
# You may now have a merge error.  You will now have to edit the offending file, in this case imu_test1.c
# Once the file has been fixed
$ git add src/test-apps/imu_test1/imu_test1.c
$ git commit -m "Fix merge"
\end{minted}

Sometimes when you do a git pull you will be thrown into a text editor
to type a merge comment. The choice of editor is controlled by an
environment variable \code{EDITOR}. On the ECE computers this defaults
to emacs. You can changed this by adding a line such as the following to
the \file{.bash\_profile} file in your home directory.

\begin{minted}{bash}
$ export EDITOR=geany
\end{minted}

By the way, to exit emacs type ctrl-x ctrl-c, to exit vi or vim type \code{:q!}
