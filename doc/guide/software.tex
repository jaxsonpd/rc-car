\label{software}
\chapter{Software}



\section{Preparation}
\label{preparation}

Do not underestimate the effort required to flash your first LED. You
require:

\begin{itemize}
\item
  A computer with \program{git} installed and a useful shell program such as
  \program{bash}. UC has a \href{http://ucmirror.canterbury.ac.nz/}{mirror}
  for a variety of Linux distributions; we recommend Ubuntu or Mint.
\item
  A working ARM toolchain (arm-none-eabi-gcc/g++ version 4.9.3 or newer).
\item
  \program{OpenOCD} V0.8.0 or later.
\item
  An ST-link programmer and 10-wire ribbon cable for programming. You
  can get the adaptor from Scott Lloyd in the SMT lab. You will need
  to make your own cable (For a grey ribbon cable, align the red
  stripe with the small arrow denoting pin 1 on the connector. For a
  rainbow ribbon cable, connect the brown wire to pin 1.). There are
  two variants of the ST-link programmer with \textbf{different
    pinouts} so you may need to customise your programming cable.
\item
  Plenty of gumption.
\end{itemize}


\subsection{Git}

Your group leader should fork the Wacky Racers project template.  This
creates your own group copy of the project on the eng-git server that
you can modify, add members, etc.

Each group member then clones the group project.


\subsection{Project forking}
\label{project-forking}

The template software is hosted on the eng-git \program{Git} server at
\href{https://eng-git.canterbury.ac.nz/wacky-racers/wacky-racers-2021}{\url{https://eng-git.canterbury.ac.nz/wacky-racers/wacky-racers-2021}}.
To fork the template:

\begin{enumerate}
\item
  Go to
  \href{https://eng-git.canterbury.ac.nz/wacky-racers/wacky-racers-2021}{\url{https://eng-git.canterbury.ac.nz/wacky-racers/wacky-racers-2021}}.
\item
  Click the `Fork' button. This will create a copy of the main repository
  for the project.
\item
  Click on the `Settings' menu then click the `Expand' button for
  `Sharing and permissions'. Change `Project Visibility' to `Private'.
\item
  Click on the `Members' menu and add group members as Developers.
\end{enumerate}


\subsection{Project cloning}
\label{project-cloning}

Once your project has been forked from the template project, each group
member needs to clone it. This makes a local copy of your project on
your computer. 

If you are using an ECE computer, it is advised that you clone the
project on to a removable USB flash drive. This will make git
operations and compilation 100 times faster than using the networked
file system.

There are two ways to clone the project. If you are impatient and do not
mind having to enter a username and password for every git pull and push
operation use:
%
\begin{minted}[breaklines]{bash}
$ git clone https://eng-git.canterbury.ac.nz/groupleader-userid/wacky-racers-2021.git wacky-racers
\end{minted}

Otherwise, set up \program{ssh-keys} and use:
%
\begin{minted}[breaklines]{bash}
$ git clone git@eng-git.canterbury.ac.nz:groupleader-userid/wacky-racers-2021.git wacky-racers
\end{minted}

You can have several different cloned copies of your project in
different directories. Sometimes if you feel that the world,
and \program{git} in particular, is against you, clone a new copy,
using:
%
\begin{minted}[breaklines]{bash}
$ git clone https://eng-git.canterbury.ac.nz/groupleader-userid/wacky-racers-2021.git wacky-racers-new
\end{minted}


\subsection{Toolchain}
\label{toolchain}

The toolchain comprises the compiler, linker, debugger, C-libraries,
and OpenOCD.

The toolchain is installed on computers in the ESL and CAE. It should
run under both Linux and Windows.  If there is a problem ask the
technical staff.

The toolchain can be downloaded for Windows, Linux, and MacOS from
from \url{https://developer.arm.com/tools-and-software/open-source-software/developer-tools/gnu-toolchain/gnu-rm/downloads}.

To install parts of the toolchain separately, see the instructions in
the following subsections.


\subsubsection{Toolchain for Linux}

First, if using Ubuntu or Mint, ensure the latest versions are
downloaded:
%
\begin{minted}{bash}
$ sudo apt update && sudo apt upgrade
\end{minted}

Then, install the compiler:
%
\begin{minted}{bash}
$ sudo apt install gcc-arm-none-eabi
\end{minted}

Install the C and C++ libraries:
%
\begin{minted}{bash}    
$ sudo apt install libnewlib-arm-none-eabi libstdc++-arm-none-eabi-newlib
\end{minted}

Install the debugger, GDB:
%
\begin{minted}{bash}    
$ sudo apt install gdb-multiarch
\end{minted}

Install OpenOCD:
%
\begin{minted}{bash}
$ sudo apt install openocd
\end{minted}

\subsubsection{Toolchain for MacOS}

For MacOS machines that have \href{https://brew.sh}{homebrew}
installed, you can use the following command:

\begin{minted}{bash}
$ brew install openocd git
$ brew cask install gcc-arm-embedded
\end{minted}


\subsection{Configuration}
\label{configuration}

Each board has different PIO definitions and requires its own
configuration information. The \wfile{boards} directory contains a
number of configurations; one for the hat and and one for racer boards
that you edit.  Each configuration directory contains three files:

\begin{itemize}
\item
  \file{board.mk} is a makefile fragment that specifies the MCU model,
  optimisation level, etc.
\item
  \file{target.h} is a C header file that defines the PIO pins and
  clock speeds.
\item
  \file{config.h} is a C header file that wraps target.h. Its purpose
  is for porting to different compilers.
\end{itemize}

\textbf{You will need to edit the target.h file for your board} and set
the definitions appropriate for your hardware. Here's an excerpt from
\file{target.h} for a hat board:

\begin{minted}{C}
/* USB  */
#define USB_VBUS_PIO PA5_PIO

/* ADC  */
#define ADC_BATTERY PB3_PIO
#define ADC_JOYSTICK_X PB2_PIO
#define ADC_JOYSTICK_Y PB1_PIO

/* IMU  */
#define IMU_INT_PIO PA0_PIO

/* LEDs  */
#define LED1_PIO PA20_PIO
#define LED2_PIO PA23_PIO
\end{minted}

\section{First program}
\label{first-program}

Your first program to test your board should only flash an LED (the
hello world equivalent for embedded systems). The key to testing new
hardware is to have many programs that only do one simple task each.

\begin{figure}
\chapter{OpenOCD}
\label{OpenOCD}

The Open On-Chip Debugger (OpenOCD) is an open-source on-chip
debugging, in-system programming, and boundary-scan testing tool. It
is able to communicate with various ARM and MIPS microprocessors via
\wikiref{JTAG}{JTAG} or \wikiref{SWD}{SWD}. It works with a number of
different \wikiref{JTAG}{JTAG} or \wikiref{SWD}{SWD}
interfaces/programmers. User interaction can be achieved via
\file{telnet/putty} or the GNU debugger (GDB).


\section{Configuration files}
\label{configuration-files}

OpenOCD needs a \wikiref{OpenOCD_configuration}{configuration file} to
specify the interface (USB or parallel port) and the target system.
Unfortunately, these change with every new release of OpenOCD.


\section{Running OpenOCD}
\label{running-openocd}

OpenOCD runs as a daemon program in the background and can be
controlled from other programs using TCP/IP sockets. This means that
you can remotely debug from another computer. The socket ports it uses
are specified in the \wikiref{OpenOCD_configuration}{OpenOCD
  configuration} file supplied when it starts. By default, OpenOCD
uses port 3333 for \wikiref{GDB}{GDB} and port 4444 for general
interaction using the \file{telnet} program.


\subsection{Communicating with OpenOCD using telnet}
\label{communicating-with-openocd-using-telnet}

If OpenOCD is running, commands can be sent to it using the telnet
program.  For example,
%
\begin{minted}{bash}
$ telnet localhost:4444
> halt
> at91sam4 info
\end{minted}
%
To exit use \code{ctrl-]} then \code{c}.  On Windows, you can use the
  \file{putty} program.


\subsection{Communicating with OpenOCD using GDB}
\label{communicating-with-openocd-using-gdb}

If OpenOCD is running, commands can be set to it with
\wikiref{GDB}{GDB}. There are two steps: connecting to OpenOCD with
the target remote command, and then sending a command with the GDB
monitor command. For example,

\begin{minted}{bash}
$ gdb
(gdb) target remote localhost:3333
(gdb) monitor flash info 0
\end{minted}

\section{OpenOCD commands}
\label{openocd-commands}

All the gory OpenOCD details can be found in the
\wikiref{Media:openocd.pdf}{OpenOCD manual}. If you are getting strange
errors see \wikiref{OpenOCD_errors}{OpenOCD errors}.

\section{Flash programming}
\label{flash-programming}

OpenOCD can program the flash program memory from a binary file.

%% \section{FTDI support}
%% \label{ftdi-support}

%% Many \wikiref{JTAG_interface}{JTAG interfaces} use chips made by Future
%% Technology Devices International (FTDI) to translate between the USB and
%% JTAG protocols. FTDI provide a library to communicate with these
%% devices; however, this is not open-source and hence cannot be
%% distributed. There is an open-source equivalent,
%% \href{http://freshmeat.net/projects/libftdi/}{libftdi}, which in turn
%% uses the open-source libusb. This can be temperamental to get working on
%% a Windows system.

%% \section{Getting OpenOCD}
%% \label{getting-openocd}

%% The developers of OpenOCD release source packages (and provide
%% subversion access) but do not provide official builds for any
%% operating system. However, there are various sites which provide
%% pre-built versions of OpenOCD.

%% \subsection{Windows}
%% \label{windows}

%% Windows installers for release versions of OpenOCD can be found
%% \href{http://www.freddiechopin.info/index.php/en/download/category/4-openocd}{here}.
%% Some development versions can be found
%% \href{http://www.freddiechopin.info/index.php/en/download/category/10-openocd-dev}{here}.
%% Unless you need a feature not present in the release version, avoid
%% getting the development versions as they can be less stable.

%% Alternatively, the
%% \href{http://www.siwawi.arubi.uni-kl.de/avr_projects/arm_projects/}{WinARM}
%% toolchain contains a version of OpenOCD, albeit one that appears (from
%% the description on the project page) to be out of date. It is also
%% possible to build your own copy of OpenOCD using either
%% \href{http://www.mingw.org/}{MinGW} or
%% \href{http://www.cygwin.com/}{Cygwin}; instructions for this are posted
%% in various places online.

%% Realistically, it's probably easier to set up a virtual machine with
%% Linux and use that instead.

%% \subsection{Linux}
%% \label{linux}

%% Many recent distributions of Linux have OpenOCD in their software
%% repositories. However, this is often out of date and (in some cases)
%% buggy. In this case it is straightforward to
%% \wikiref{Building_OpenOCD_under_Linux}{build the latest version}.

%% \subsection{Mac OSX}
%% \label{mac-osx}

%% The easiest way to install OpenOCD on Mac OSX is to use
%% \href{http://brew.sh/}{Homebrew}. If you wish to use JTAG with an
%% adaptor based on a FTDI chip (for example, the UCECE USB to JTAG
%% adaptor or a Bus Blaster), make sure you include libftdi support by
%% installing with the command:
%% %
%% \begin{minted}{bash}
%% $ brew install openocd --enable-ft2232_libftdi
%% \end{minted}

\section{System information}
\label{system-information}


The openocd command \code{at91sam4 info}, see
\hyperref[communicating-with-openocd-using-telnet]{communicating with
  OpenOCD using telnet}, outputs useful information about the state of
the SAM4S.  For example:
%
\begin{verbatim}
    CKGR_MOR: [0x400e0420] -> 0x01006409
	    MOSCXTEN:     1 [0x0001] (main xtal enabled: YES)
	    MOSCXTBY:     0 [0x0000] (main osc bypass: NO)
	    MOSCRCEN:     1 [0x0001] (onchip RC-OSC enabled: YES)
	     MOSCRCF:     0 [0x0000] (onchip RC-OSC freq: 4 MHz)
	    MOSCXTST:   100 [0x0064] (startup clks, time= 3051.757812 uSecs)
	     MOSCSEL:     1 [0x0001] (mainosc source: external xtal)
	       CFDEN:     0 [0x0000] (clock failure enabled: NO)
   CKGR_MCFR: [0x400e0424] -> 0x000117ac
	    MAINFRDY:     1 [0x0001] (main ready: YES)
	       MAINF:  6060 [0x17ac] (12.411 Mhz (32.768khz slowclk)
  CKGR_PLLAR: [0x400e0428] -> 0x08133f01
	        DIVA:     1 [0x0001]
	        MULA:    19 [0x0013]
	PLLA Freq: 248.218 MHz
   CKGR_UCKR: [0x400e041c] -> 0x00000000
    PMC_FSMR: [0x400e0470] -> 0x00000000
    PMC_FSPR: [0x400e0474] -> 0x00000000
     PMC_IMR: [0x400e046c] -> 0x00000000
    PMC_MCKR: [0x400e0430] -> 0x00000012
	         CSS:     2 [0x0002] plla (248.218 Mhz)
	        PRES:     1 [0x0001] (clock/2)
		Result CPU Freq: 124.109
    PMC_PCK0: [0x400e0440] -> 0x00000000
    PMC_PCK1: [0x400e0444] -> 0x00000000
    PMC_PCK2: [0x400e0448] -> 0x00000000
    PMC_PCSR: [0x400e0418] -> 0x41004000
    PMC_SCSR: [0x400e0408] -> 0x00000001
      PMC_SR: [0x400e0468] -> 0x0003000f
 CHIPID_CIDR: [0x400e0740] -> 0x289c0ae0
	     Version:     0 [0x0000]
	       EPROC:     7 [0x0007] Cortex-M4
	     NVPSIZE:    10 [0x000a] 512K bytes
	    NVPSIZE2:     0 [0x0000] none
	    SRAMSIZE:    12 [0x000c] 128K Bytes
	        ARCH:   137 [0x0089] ATSAM3S/SAM4S xB Series (64-pin version)
	      NVPTYP:     2 [0x0002] embedded flash memory
	       EXTID:     0 [0x0000] (exists: NO)
 CHIPID_EXID: [0x400e0744] -> 0x00000000
   rc-osc: 4.000 MHz
  mainosc: 12.411 MHz
     plla: 248.218 MHz
 cpu-freq: 124.109 MHz
mclk-freq: 124.109 MHz
 UniqueId: 0x53343100 0x50524a46 0x36303430 0x30363030
\end{verbatim}
%
Note, the reported clock frequencies are not accurate.


\section{Errors}
\label{openocd_errors}

\verb+Error: libusb_open() failed with LIBUSB_ERROR_ACCESS+ On Linux
you need to give permissions to access the USB.  You can do this by
finding the file \file{99-openocd.rules} that comes with the OpenOCD
distribution and then:
%
\begin{minted}{bash}
  $ sudo cp 99-openocd.rules /etc/udev/udev.rules
  $ sudo udevadm control --reload
\end{minted}
%
Finally, you need to reconnect the ST-link device.


See also \wikiref{OpenOCD_errors}{OpenOCD errors}.


\section{Programming without GDB}

OpenOCD can program the SAM4S without using GDB.  You will need to
kill any currently running instance of OpenOCD and then use:
%
\begin{minted}{bash}
  $ openocd -f path-to-sam4s-stlink.cfg -c "program program.bin verify reset exit"
\end{minted}
%
Here \file{program.bin} is the name of the program to load.

\caption{How OpenOCD interacts with the debugger and the SAM4S.}
\label{fig:openocd diagram}
\end{figure}


\subsection{OpenOCD}
\label{openocd}

\program{OpenOCD} is used to program the SAM4S, see \reffig{openocd diagram}.

For this assignment, we use a ST-link programmer to connect to the
SAM4S using serial wire debug (SWD). This connects to your board with
a 10-wire ribbon cable and an IDC connector.

\begin{enumerate}
\item
  Before you start, disconnect the battery and other cables from your
  PCB.
\item
  Connect a 10-wire ribbon cable from the ST-link programmer to the
  programming header on your PCB. This will provide 3.3 V to your
  board so your green power LED should light.
\item
  Open a \textbf{new terminal window, e.g., bash} and
  start \program{OpenOCD}.
\end{enumerate}

\begin{minted}{bash}
$ cd wacky-racers
$ openocd -f src/mat91lib/sam4s/scripts/sam4s_stlink.cfg
\end{minted}

All going well, the last line output from \program{OpenOCD} should be:

\begin{verbatim}
Info : sam4.cpu: hardware has 6 breakpoints, 4 watchpoints
\end{verbatim}

Congrats if you get this! It means you have correctly soldered your
SAM4S. If not, do not despair and do not remove your SAM4S. Instead,
see \protect\hyperref[troubleshooting]{troubleshooting}.


\subsection{LED flash program}
\label{led-flash-program}

For your first program, use
\wfile{test-apps/ledflash1/ledflash1.c}. The macros
\code{LED1_PIO} and \code{LED2_PIO} need to be defined in
\file{target.h} (see
\protect\hyperref[configuration]{configuration}).

\inputminted{C}{../../src/test-apps/ledflash1/ledflash1.c}

\subsection{Compilation}
\label{compilation}

Due to the many files required, compilation is performed using
makefiles.

The demo test programs are generic and you need to specify which board
you are compiling them for. The board configuration file can be chosen
dynamically by defining the environment variable \code{BOARD}. For
example:
%
\begin{minted}{bash}
$ cd src/test-apps/ledflash1
$ BOARD=racer make
\end{minted}

If all goes well, you should see at the end:
%
\begin{verbatim}
   text    data     bss     dec     hex filename
  11348	   2416	    176	  13940	   3674	ledflash1.bin
\end{verbatim}

To avoid having to specify the environment variable \code{BOARD}, you
can define it for the rest of your session using:
%
\begin{minted}{bash}
$ export BOARD=racer
\end{minted}
%
and then just use:
%
\begin{minted}{bash}
$ make
\end{minted}

\subsection{Booting from flash memory}
\label{booting-from-flash-memory}

By default the SAM4S runs a bootloader program stored in ROM. The SAM4S
needs to be configured to run your application in flash memory.

If \program{OpenOCD} is running you can do this with:

\begin{minted}{bash}
$ make bootflash
\end{minted}

Unless you force a complete erasure of the SAM4S flash memory by
connecting the \pin{ERASE} pin to 3.3 V, you will not need to repeat
this command.

\subsection{Programming}
\label{programming}

If \program{OpenOCD} is running you can store your program in the flash
memory of the SAM4S using:

\begin{minted}{bash}
$ make program
\end{minted}

When this finishes, one of your LEDs should flash. If so, congrats! If
not, see \protect\hyperref[troubleshooting]{troubleshooting}.

To reset your SAM4S, you can use:
%
\begin{minted}{bash}
$ make reset
\end{minted}

\section{USB interfacing}
\label{usb-interfacing}

To help debug your programs, it is wise to use \wikiref{USB CDC}{USB
  CDC}.  This is a serial protocol for USB.  With some magic, the
stdin, stdout, and stderr streams can sent over USB.  For example,
here's an example program
\wfile{test-apps/usb_serial_test1/usb_serial_test1.c}.

\inputminted{C}{../../src/test-apps/usb_serial_test1/usb_serial_test1.c}

Note, the string \code{"/dev/usb_tty"} is used to name the USB serial
device on the SAM4S so that it can be used by the C stdio function
\code{freopen}\footnote{When running a program with an operating
  system, this is all set up before \code{main} is called.}.

To get this program to work you need to compile it and program the
SAM4S using:
%
\begin{minted}{bash}
$ cd wacky-racers/src/test-apps/usb_serial_test1
$ make program
\end{minted}

You then need to connect your computer to the USB connector on your PCB.
If you are running Linux, run:
%
\begin{minted}{bash}
$ dmesg
\end{minted}

This should say something like:
%
\begin{verbatim}
Apr 30 11:03:50 thing4 kernel: [52704.481352] usb 2-3.3: New USB device found, idVendor=03eb, idProduct=6202
Apr 30 11:03:50 thing4 kernel: [52704.481357] usb 2-3.3: New USB device strings: Mfr=1, Product=2, SerialNumber=3
Apr 30 11:03:50 thing4 kernel: [52704.482060] cdc_acm 2-3.3:1.0: ttyACM0: USB ACM device
\end{verbatim}

Congrats if you see \texttt{ttyACM0:\ USB\ ACM\ device}!  If not, see
\wikiref{USB_debugging}{USB debugging}.

You can now run a \wikiref{Serial_terminal_applications}{serial
  terminal program}. For example, on Linux:
%
\begin{minted}{bash}
$ gtkterm -p /dev/ttyACM0
\end{minted}

All going well, this will repeatedly print 'Hello world'.

If run Linux and get an error 'device is busy', it is likely that the
ModemManager program has automatically connected to your device on the
sly. This program should be disabled on the ECE computers. For more
about this and using other operating systems, see \wikiref{USB
CDC}{USB CDC}.

\section{Test programs}
\label{test-programs}

There are a number of test programs in the directory
\wfile{test-apps}.  Where possible these are written to be
independent of the target board using configuration files (see
\protect\hyperref[configuration]{configuration}).

\subsection{PWM test}
\label{pwm-test}

The program \wfile{test-apps/pwm_test2/pwm_test2.c} provides an
example of driving PWM signals.

\inputminted{C}{../../src/test-apps/pwm_test2/pwm_test2.c}

Notes:
%
\begin{enumerate}
\item
  This is for a different H-bridge module that requires two PWM signals
  and forward/reverse signals. You will need to generate four PWM
  signals or be clever with two PWM signals.
\item
  The \code{pwm_cfg_t} structure configures the frequency, duty
  cycle and alignment of the output PWM.
\item The frequency is likely to be too high for your motor.
\item \code{pwm_channels_start} is used to start the PWM channels simultaneously.  
\end{enumerate}

The most likely problem is that you have not used a PIO pin that can
be driven as a PWM output. The SAM4S can generate four independent
hardware PWM signals. See \wfile{mat91lib/pwm/pwm.c} for a list of
supported PIO pins. Note, \pin{PA16}, \pin{PA30}, and \pin{PB13} are
different options for PWM2.

To drive the motors you will need to use a bench power supply. Start
with the current limit set at 100\,mA maximum in case there are any
board shorts.  When all is well, you can increase the current limit;
you will need at least 1\,A.

\subsection{IMU test}
\label{imu-test}

The MPU9250 IMU connects to the SAM4S using the I2C bus (aka TWI bus).
The program \wfile{test-apps/imu_test1/imu_test1.c} provides an
example of using the MPU9250 IMU.  All going well, this prints three
16-bit acceleration values per line to USB CDC. Tip your board over,
and the the third (z-axis) value should go negative since this
measures the effect of gravity on a little mass inside the IMU pulling
on a spring.

If you get `ERROR: can't find MPU9250!', the main reasons are:

\begin{enumerate}
\item
  You have specified the incorrect address.  Use 0x68 for
  \code{MPU_ADDRESS} in \file{target.h} if the AD0 pin is connected to
  ground otherwise use 0x69.
\item
  You are using TWI1. The \pin{PB4} and \pin{PB5} pins used by TWI1
  default to JTAG pins. See
  \protect\hyperref[disabling-jtag-pins]{disabling JTAG pins}.
\end{enumerate}
%
See also \protect\hyperref[checking-IMU]{IMU checking}.

Other problems:
%
\begin{enumerate}
\item If you reset the SAM4S in the middle of a transaction with the
IMU, the IMU gets confused and holds the \pin{TWD/SDA} line low. This
requires recycling of the power or sending out some dummy clocks on
the \pin{TWCK/SCL} signal.
\end{enumerate}


\subsection{Radio test}
\label{radio-test}

The program \wfile{test-apps/radio_tx_test1/radio_tx_test1.c} provides
an example of using the radio as a transmitter.

The companion program
\wfile{test-apps/radio_rx_test1/radio_rx_test1.c} provides an example
of using the radio as a receiver.

Notes:
%
\begin{enumerate}
\item
  Both programs must use the same RF channel and the same address. Some
  RF channels are better than others since some overlap with WiFi
  and Bluetooth. The address is used to distinguish devices
  operating on the same channel. Note, the transmitter expects an
  acknowledge from a receiver on the same address and channel.
\item
  The radio `write` method blocks waiting for an auto-acknowledgement
  from the receiver device. This acknowledgement is performed in
  hardware. If no acknowledgement is received, it retries for up to 15
  times. The auto-acknowledgement and number of retries can be
  configured in software.

\item If the program hangs in the \code{panic} loop, there is no
  response from the radio module, check SPI connections and see
  \hyperref[debugging_Spi]{SPI debugging}.
  
\end{enumerate}

If you cannot communicate between your hat and racer boards, try
communicating with the radio test modules Scott Lloyd has in the SMT
lab.

\section{Your hat/racer program}
\label{your-hatracer-program}

We recommend that build your programs incrementally and that you poll
your devices with a paced loop and not use interrupts.  It is a good
idea to disable the \hyperref[watchdog-timer]{watchdog timer} until
you have robust code.

\section{Howtos}
\label{howtos}

\subsection{PIO pins}

mat91lib provides efficient PIO abstraction routines in
\wfile{mat91lib/sam4s/pio.h}.  Each pin can be configured as follows:
%
\begin{minted}{C}
    PIO_INPUT,              /* Configure as input pin.  */
    PIO_PULLUP,             /* Configure as input pin with pullup.  */
    PIO_PULLDOWN,           /* Configure as input pin with pulldown.  */
    PIO_OUTPUT_LOW,         /* Configure as output, initially low.  */
    PIO_OUTPUT_HIGH,        /* Configure as output, initially high.  */
    PIO_PERIPH_A,           /* Configure as peripheral A.  */
    PIO_PERIPH_A_PULLUP,    /* Configure as peripheral A with pullup.  */
    PIO_PERIPH_B,           /* Configure as peripheral B.  */
    PIO_PERIPH_B_PULLUP,    /* Configure as peripheral B with pullup.  */
    PIO_PERIPH_C,           /* Configure as peripheral C.  */
    PIO_PERIPH_C_PULLUP     /* Configure as peripheral C with pullup.  */
\end{minted}

Here's an example:
%
\begin{minted}{C}
  #include "pio.h"

  // Configure PA0 as an output and set default state to low.
  pio_config_set (PIO_PA0, PIO_OUTPUT_LOW);

  // Set PA0 high.
  pio_output_high (PIO_PA0);

  // Set PA0 low.  
  pio_output_low (PIO_PA0);

  // Set PA0 to value.
  pio_output_set (PIO_PA0, value);  

  // Toggle PA0.  
  pio_output_toggle (PIO_PA0);    

  // Reonfigure PA0 as an output connected to peripheral A.
  pio_config_set (PIO_PA0, PIO_PERIPH_A);

  // Reonfigure PA0 as an input with pullup enabled.
  pio_config_set (PIO_PA0, PIO_INPUT_PULLUP);

  // Read state of PIO pin.
  result = pio_input_get (PIO_PA0);
\end{minted}

Note, you can reconfigure a PIO pin on the fly.  For example, you may
want the pin to be driven by the PWM peripheral and then at some stage
forced low.  To do this, use \code{pwm\_config\_set}.


\subsection{Delaying}

mat91lib provides a macro \code{DELAY_US} in
\wfile{mat91lib/delay.h} for a busy-wait delay in microseconds
(this can be a floating point value).  The CPU just spins for a
precomputed number of clock cycles.  The argument should be a constant
so the compiler can compute the number of clock cycles.  This function
needs to be compiled with optimisation.

mat91lib also provides a function \code{delay_ms} for a busy-wait
delay in milliseconds (this must be an integer).  All this function
does is call \code{DELAY_US (1000)} the required number of times.

An example program is \wfile{test-apps/delay_test1/delay_test1.c}.


\subsection{Reading from ADC}
\label{ADC}

\wfile{test-apps/adc_usb_serial_test2/adc_usb_serial_test2.c} shows
how to read from two multiplexed ADC channels.  For more details
see \wfile{mat91lib/adc/adc.h}.

\inputminted{C}{../../src/test-apps/adc_usb_serial_test2/adc_usb_serial_test2.c}

The ADC can be also set up to stream data continuously; this is a lot
more complicated.


\subsection{Reading from pushbutton}
\label{pushbutton}

\wfile{test-apps/button_test2/button_test2.c} shows the use of a
simple button driver to read a pushbutton.  This driver does button
debouncing and state-transition detection.  For more details see
\wfile{mmculib/button/button.h}.

\inputminted{C}{../../src/test-apps/button_test2/button_test2.c}



\subsection{Disabling JTAG pins}
\label{disabling-jtag-pins}

By default \pin{PB4} and \pin{PB5} are configured as JTAG pins. You can turn
them into PIO pins or use them for TWI1 using:
%
\begin{minted}{C}
#include "mcu.h"

void main (void)
{
    mcu_jtag_disable ();
}
\end{minted}

\subsection{Watchdog timer}
\label{watchdog-timer}

The watchdog timer is useful for resetting the SAM4S if it hangs in a
loop.  It is disabled by default but can be enabled using:
%
\begin{minted}{C}
#include "mcu.h"

void main (void)
{
    mcu_watchdog_enable ();
   
    while (1)
    {
         /* Do your stuff here.  */

         mcu_watchdog_reset ();
    }
}
\end{minted}

\section{Under the bonnet}
\label{under-the-bonnet}

\file{mmculib} is a library of C drivers, mostly for performing high-level I/O.
It is written to be microcontroller neutral.

\file{mat91lib} is a library of C drivers specifically for interfacing
with the peripherals of Atmel AT91 microcontrollers such as the Atmel
SAM4S. It provides the hardware abstraction layer.

The building is controlled by \wfile{mat91lib/mat91lib.mk}. This is a
makefile fragment loaded by \wfile{mmculib/mmculib.mk}.
\file{wacky-reacers/src/mat91lib/mat91lib.mk} loads other makefile fragments for each peripheral or driver required. It also automatically generates
dependency files for the gazillions of other files that are required
to make things work.


\textbf{Please do not edit the files in the mat91lib, mmculib, and
wackylib} directories since this can lead to merge problems in the
future. If you find a bug or would like additional functionality let MPH
or one of the TAs know.
