\chapter{PCB assembly}


\section{SMT lab induction}

You cannot use the SMT lab unless you have performed a lab induction.


\section{Solder pasting}

Solder paste needs to be applied to each pad.  Usually this is applied
using a stencil but for small boards, it can be applied using a
syringe.  The solder paste must be fresh.


\section{Component placement}

The components are placed onto the pasted pads using a pick-and-place
machine.  The key is to get the orientation of chips, diodes, and LEDs
correct.

It is easy to orient the MCU incorrectly.  The big dot \textbf{does
  not mark pin 1}\footnote{I think they add this to sell more chips.}.


\mtodo{Show image of MCU and where pin 1 is.}


\section{Reflow oven}

When the board has been populated, the PCB is placed in the reflow
oven.  At the end of the process, remove your PCB using oven gloves.
You do not need to clean the board.


\section{Visual inspection}

Look for solder bridges, especially between IC pins.  These can be
removed with a fine-tipped soldering iron.

\mtodo{Show photo of solder bridge}
