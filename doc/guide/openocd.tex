\chapter{OpenOCD}
\label{OpenOCD}

The Open On-Chip Debugger (OpenOCD) is an open-source on-chip
debugging, in-system programming, and boundary-scan testing tool. It
is able to communicate with various ARM and MIPS microprocessors via
\wikiref{JTAG}{JTAG} or \wikiref{SWD}{SWD}. It works with a number of
different \wikiref{JTAG}{JTAG} or \wikiref{SWD}{SWD}
interfaces/programmers. User interaction can be achieved via
\file{telnet/putty} or the GNU debugger (GDB).


\section{Configuration files}
\label{configuration-files}

OpenOCD needs a \wikiref{OpenOCD_configuration}{configuration file} to
specify the interface (USB or parallel port) and the target system.
Unfortunately, these change with every new release of OpenOCD.


\section{Running OpenOCD}
\label{running-openocd}

OpenOCD runs as a daemon program in the background and can be
controlled from other programs using TCP/IP sockets. This means that
you can remotely debug from another computer. The socket ports it uses
are specified in the \wikiref{OpenOCD_configuration}{OpenOCD
  configuration} file supplied when it starts. By default, OpenOCD
uses port 3333 for \wikiref{GDB}{GDB} and port 4444 for general
interaction using the \file{telnet} program.


\subsection{Communicating with OpenOCD using telnet}
\label{communicating-with-openocd-using-telnet}

If OpenOCD is running, commands can be sent to it using the telnet
program.  For example,
%
\begin{minted}{bash}
$ telnet localhost:4444
> halt
> at91sam4 info
\end{minted}
%
To exit use \code{ctrl-]} then \code{c}.  On Windows, you can use the
  \file{putty} program.


\subsection{Communicating with OpenOCD using GDB}
\label{communicating-with-openocd-using-gdb}

If OpenOCD is running, commands can be set to it with
\wikiref{GDB}{GDB}. There are two steps: connecting to OpenOCD with
the target remote command, and then sending a command with the GDB
monitor command. For example,

\begin{minted}{bash}
$ gdb
(gdb) target remote localhost:3333
(gdb) monitor flash info 0
\end{minted}

\section{OpenOCD commands}
\label{openocd-commands}

All the gory OpenOCD details can be found in the
\wikiref{Media:openocd.pdf}{OpenOCD manual}. If you are getting strange
errors see \wikiref{OpenOCD_errors}{OpenOCD errors}.

\section{Flash programming}
\label{flash-programming}

OpenOCD can program the flash program memory from a binary file.

%% \section{FTDI support}
%% \label{ftdi-support}

%% Many \wikiref{JTAG_interface}{JTAG interfaces} use chips made by Future
%% Technology Devices International (FTDI) to translate between the USB and
%% JTAG protocols. FTDI provide a library to communicate with these
%% devices; however, this is not open-source and hence cannot be
%% distributed. There is an open-source equivalent,
%% \href{http://freshmeat.net/projects/libftdi/}{libftdi}, which in turn
%% uses the open-source libusb. This can be temperamental to get working on
%% a Windows system.

%% \section{Getting OpenOCD}
%% \label{getting-openocd}

%% The developers of OpenOCD release source packages (and provide
%% subversion access) but do not provide official builds for any
%% operating system. However, there are various sites which provide
%% pre-built versions of OpenOCD.

%% \subsection{Windows}
%% \label{windows}

%% Windows installers for release versions of OpenOCD can be found
%% \href{http://www.freddiechopin.info/index.php/en/download/category/4-openocd}{here}.
%% Some development versions can be found
%% \href{http://www.freddiechopin.info/index.php/en/download/category/10-openocd-dev}{here}.
%% Unless you need a feature not present in the release version, avoid
%% getting the development versions as they can be less stable.

%% Alternatively, the
%% \href{http://www.siwawi.arubi.uni-kl.de/avr_projects/arm_projects/}{WinARM}
%% toolchain contains a version of OpenOCD, albeit one that appears (from
%% the description on the project page) to be out of date. It is also
%% possible to build your own copy of OpenOCD using either
%% \href{http://www.mingw.org/}{MinGW} or
%% \href{http://www.cygwin.com/}{Cygwin}; instructions for this are posted
%% in various places online.

%% Realistically, it's probably easier to set up a virtual machine with
%% Linux and use that instead.

%% \subsection{Linux}
%% \label{linux}

%% Many recent distributions of Linux have OpenOCD in their software
%% repositories. However, this is often out of date and (in some cases)
%% buggy. In this case it is straightforward to
%% \wikiref{Building_OpenOCD_under_Linux}{build the latest version}.

%% \subsection{Mac OSX}
%% \label{mac-osx}

%% The easiest way to install OpenOCD on Mac OSX is to use
%% \href{http://brew.sh/}{Homebrew}. If you wish to use JTAG with an
%% adaptor based on a FTDI chip (for example, the UCECE USB to JTAG
%% adaptor or a Bus Blaster), make sure you include libftdi support by
%% installing with the command:
%% %
%% \begin{minted}{bash}
%% $ brew install openocd --enable-ft2232_libftdi
%% \end{minted}

\section{System information}
\label{system-information}


The openocd command \code{at91sam4 info}, see
\hyperref[communicating-with-openocd-using-telnet]{communicating with
  OpenOCD using telnet}, outputs useful information about the state of
the SAM4S.  For example:
%
\begin{verbatim}
    CKGR_MOR: [0x400e0420] -> 0x01006409
	    MOSCXTEN:     1 [0x0001] (main xtal enabled: YES)
	    MOSCXTBY:     0 [0x0000] (main osc bypass: NO)
	    MOSCRCEN:     1 [0x0001] (onchip RC-OSC enabled: YES)
	     MOSCRCF:     0 [0x0000] (onchip RC-OSC freq: 4 MHz)
	    MOSCXTST:   100 [0x0064] (startup clks, time= 3051.757812 uSecs)
	     MOSCSEL:     1 [0x0001] (mainosc source: external xtal)
	       CFDEN:     0 [0x0000] (clock failure enabled: NO)
   CKGR_MCFR: [0x400e0424] -> 0x000117ac
	    MAINFRDY:     1 [0x0001] (main ready: YES)
	       MAINF:  6060 [0x17ac] (12.411 Mhz (32.768khz slowclk)
  CKGR_PLLAR: [0x400e0428] -> 0x08133f01
	        DIVA:     1 [0x0001]
	        MULA:    19 [0x0013]
	PLLA Freq: 248.218 MHz
   CKGR_UCKR: [0x400e041c] -> 0x00000000
    PMC_FSMR: [0x400e0470] -> 0x00000000
    PMC_FSPR: [0x400e0474] -> 0x00000000
     PMC_IMR: [0x400e046c] -> 0x00000000
    PMC_MCKR: [0x400e0430] -> 0x00000012
	         CSS:     2 [0x0002] plla (248.218 Mhz)
	        PRES:     1 [0x0001] (clock/2)
		Result CPU Freq: 124.109
    PMC_PCK0: [0x400e0440] -> 0x00000000
    PMC_PCK1: [0x400e0444] -> 0x00000000
    PMC_PCK2: [0x400e0448] -> 0x00000000
    PMC_PCSR: [0x400e0418] -> 0x41004000
    PMC_SCSR: [0x400e0408] -> 0x00000001
      PMC_SR: [0x400e0468] -> 0x0003000f
 CHIPID_CIDR: [0x400e0740] -> 0x289c0ae0
	     Version:     0 [0x0000]
	       EPROC:     7 [0x0007] Cortex-M4
	     NVPSIZE:    10 [0x000a] 512K bytes
	    NVPSIZE2:     0 [0x0000] none
	    SRAMSIZE:    12 [0x000c] 128K Bytes
	        ARCH:   137 [0x0089] ATSAM3S/SAM4S xB Series (64-pin version)
	      NVPTYP:     2 [0x0002] embedded flash memory
	       EXTID:     0 [0x0000] (exists: NO)
 CHIPID_EXID: [0x400e0744] -> 0x00000000
   rc-osc: 4.000 MHz
  mainosc: 12.411 MHz
     plla: 248.218 MHz
 cpu-freq: 124.109 MHz
mclk-freq: 124.109 MHz
 UniqueId: 0x53343100 0x50524a46 0x36303430 0x30363030
\end{verbatim}
%
Note, the reported clock frequencies are not accurate.


\section{Errors}
\label{openocd_errors}

\verb+Error: libusb_open() failed with LIBUSB_ERROR_ACCESS+ On Linux
you need to give permissions to access the USB.  You can do this by
finding the file \file{99-openocd.rules} that comes with the OpenOCD
distribution and then:
%
\begin{minted}{bash}
  $ sudo cp 99-openocd.rules /etc/udev/udev.rules
  $ sudo udevadm control --reload
\end{minted}
%
Finally, you need to reconnect the ST-link device.


See also \wikiref{OpenOCD_errors}{OpenOCD errors}.


\section{Programming without GDB}

OpenOCD can program the SAM4S without using GDB.  You will need to
kill any currently running instance of OpenOCD and then use:
%
\begin{minted}{bash}
  $ openocd -f path-to-sam4s-stlink.cfg -c "program program.bin verify reset exit"
\end{minted}
%
Here \file{program.bin} is the name of the program to load.
