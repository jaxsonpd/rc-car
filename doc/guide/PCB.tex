\chapter{PCB layout}


\section{Introduction}
\label{PCB_introduction}

An hour spent checking your schematic will avoid many hours of testing
and rework grief!




\section{PCB recommendations}
\label{pcb-recommendations}

\subsection{Placement}
\label{placement}

\begin{enumerate}
\item
  Keep small signal analogue components (radio) well away from digital
  electronics and power electronics.
\item
  Place local decoupling capacitors to minimise the loop area.
\item
  Keep the crystal close to the MCU.
\item
  Place switches so they can be pushed.
\item
  Place LEDs so they can be seen.
\item
  Place USB connector so it can be connected to.
\item
  Place connectors on edge with wires going away from the board.
\end{enumerate}


\subsection{Check list}\label{check-list}

\begin{enumerate}
\item
  No planes under or close to the radio antenna; otherwise the radio
  range is limited.
\item
  The SPI signals for the radio are connected to the correct MCU pins.
\item
  The PWM signals for the motor are connected to the correct MCU pins.
  Note, the PWMLx and PWMHx pins are complementary and cannot be driven
  independently.
\item
  All the MCU VDD pins need to be powered.
\item
  All the MCU GND pins need to be powered.
\item
  Avoid connecting to \pin{PB4} and \pin{PB5} (say for TWI1).  If you
  do you will need to disable JTAG in software.
\end{enumerate}

