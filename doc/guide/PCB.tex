\chapter{PCB layout}


\section{Introduction}
\label{PCB_introduction}

Before you lay out your PCB, an hour spent checking your schematic
will avoid many hours of testing and rework grief!


\label{pcb-recommendations}

\section{Placement}
\label{placement}

\begin{enumerate}
\item
  Keep small signal analogue components (radio) well away from digital
  electronics and power electronics.
\item
  Place local decoupling capacitors to minimise the loop area.
\item
  Place bulk capacitors close to where power comes from.
\item
  Keep the crystal close to the MCU.
\item
  Place switches so they can be pushed.
\item
  Place LEDs so they can be seen.
\item
  Place USB connector so it can be connected to.
\item
  Place connectors on edge with wires going away from the board.
\end{enumerate}


\section{Power supplies}

\begin{enumerate}
  
\item Use planes for power distribution.

\item If cannot use power plane for power connection, make trace as
  wide\footnote{To reduce inductance and resistance.} as possible.

\item Do not put splits in planes\footnote{Unless you know what you
  are doing.}.

\item Keep planes away from the radio antenna; otherwise the radio
  range is limited.

\item Before `pouring' a polygon to create a power or ground plane
  connect the nets with tracks and vias.
\end{enumerate}


\section{Signal traces}

\begin{enumerate}
  
\item
  Use microstrips for any signal above 50\,kHz.

\item
  Keep signal traces apart to reduce crosstalk (3-W rule).

\item
  Avoid signal traces jumping layers (especially for signals above
  50\,kHz).  If you do, use vias close to ends of traces.

\item
  Use tented\footnote{These have a layer of insulating solder resist.}
  vias if under components with metal pads.
  
\end{enumerate}


\section{Recommended layouts}

For any switching power supply, sensitive analogue, or BGA fanout, the
chip manufacturer usually provides a recommended layout.

\section{Check list}
\label{PCB-check-list}

\begin{enumerate}
\item Check your schematic.  Then get someone else to check your
  schematic.  Query every component.

\item Perform a DRC (Design Rule Check)] on the PCB.
  
\item Add PCB test points, test points, and more test points.  '''You
  never have enough''' for a prototype.  You have been warned!  If you
  don't believe me, compare the size of an oscilloscope probe to a
  surface mount IC pin.
  
\item Label the test points on the silk screen with meaningful names.
  
\item Add ground test points that you can clip an oscilloscope ground
  lead to.  Keep these away from other test points to avoid shorting
  with the oscilloscope ground clip.
  
\item Check mounting holes.
  
\item Check component footprints, especially connectors, voltage
  regulators, and surface mount transistors by placing the parts on a
  print-out of the PCB design.
  
\item Ensure that power supply traces are wide as possible, preferably
  planes, to reduce their inductance.
  
\item Ensure that high current traces are wide as possible to reduce their resistance.
  
\item Do not connect IC pins directly between the pads since it is not
  possible to tell if a solder blob is accidental or deliberate.

\item Insert vias to ground plane to ensure that all parts are connected.  Do not leave unconnected copper.
  
\item If not using plated through vias do not place vias under
  components.
  
\item If placing vias under components with exposed metal pads, make
  sure the vias are "tented" (covered in solder mask) to prevent
  unexpected shorts.
  
\item Stitch power or ground planes where vias or tracks cut them up.

\item If a track is of width W it should be separated from the next
  track by a width 2W to reduce crosstalk.  This is the 3-W rule since
  the track centres are 3W.  An exception is for differential signals;
  these should be routed close together with a uniform spacing.

\item Do not run a microstrip traces right on the edge of the PCB.  If
  track is of width W it should be at least W from the edge to reduce
  electric field fringing effects.
  
\item Use large traces for pads on connectors that may be subject to
  mechanical forces (power jacks, pluggable terminals, etc.) to
  prevent trace cracking.
  
\item If you have high voltages check clearances.
  
\item Check pad to track clearances to avoid potential solder bridges.
  
\item Check that it is possible to top-solder through-hole components
  on non-plated-through boards.
  
\item Ensure like signals are grouped, data lines, address lines, control lines, analogue signals, etc.
  
\item Check the thermal path for parts that get hot.
  
\item Run tracks over reference (power or ground) planes to create a
  microstrip trace; avoid tracks running over cuts in a reference
  plane.
  
\item Check that hole diameters are larger than the pin diameters for
  connectors and through hole. 25\% over the lead diameters is
  typical.  Remember to look at the diagonal dimension for square pins
  or your holes may be too small.
  
\item Check the drill file report to make sure the hole sizes are what
  you would expect. There should be no holes with sizes of zero, no
  weird sizes, no super large sizes unless required by the design.
  
\item Check that connections of thermal vias have clearance to other
  traces and pads.

\item Pay close attention to clearances on internal planes of
  multilayer boards; shorts on these planes can only be removed with
  precision drilling.
  
\item Number pins on connectors, big ICs, selection jumpers, etc. For
  connectors, number enough pins that pin ordering is obvious. For
  large ICs consider adding tickmarks every 5 pins.
  
\item Make the pad for pin 1 of every IC square. Also consider putting
  a silkscreen dot next to pin 1.

\item Leave at least 20-50 mils (0.5-1.3 mm) clearance between components and the edge of the board. This makes it less likely that inaccuracies in board fabrication will cause part of a pad to get chopped off.
  
\item Check for component clearances from enclosure including heights.
  
\item Look at each net individually to ensure that it doesn’t take an
  overly long path around the board. Many layout tools have a
  highlight net feature that makes this a fast process.
  
\item Don't let the silkscreen overlap pads. Try to avoid having
  silkscreen overlapping via holes or areas of the board that will be
  routed away (large holes, slots, tabs, etc.), since legibility will
  be poor.
  
\item If ICs are being soldered with solder paste ensure that there is
  a solder mask between pads.
 \end{enumerate}

