\chapter{EMC}


Electromagnetic emission (EMI) occurs with changing electric and
magnetic fields.  A good PCB design should have good electromagnetic
compatibility (EMC); this means that it has low EMI and that is has
low susceptibility to changing electric and magnetic fields.


\section{Electromagnetic coupling}

It is worth recapping the fundamental equations.  A changing aggressor
current, $i_a(t)$, flowing around a loop induces a
voltage\footnote{This voltage magically appears around a loop and does
  not obey Kirchhoff's voltage law.  It will mostly `appear' across
  the highest resistance in the loop.} in a victim loop according to
Faraday's law,
%
\begin{equation}
  v_v(t) = M \frac{\ud i_a(t)}{\ud t},
\end{equation}
%
where $M$ is the mutual inductance.  The mutual inductance depends on
the areas of the aggressor and victim loops and their orientation.  It
is minimised by using microstrips.

A changing aggressor voltage induces a current in a victim circuit
according to
%
\begin{equation}
  i_v(t) = C \frac{\ud v_a(t)}{\ud t},
\end{equation}
%
where $C$ is the mutual capacitance that depends on the separation
between the traces and the distance the traces run beside each other.
From Ohm's law, the victim current will produce a voltage $i_v(t) R$
across a resistance $R$.  Thus low resistance circuits are more immune
to capacitive coupling.
