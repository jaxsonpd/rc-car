\chapter{Debugging}
\label{debugging}

Your SAM4S can be debugged using \hyperref[OpenOCD]{OpenOCD} in
conjunction with GDB.  However, debugging embedded systems is
difficult due to asynchronous behaviour.  Moreover, when optimising, a
compiler will rearrange (and even remove) code and hold variables in
registers.

\section{Command-line debugging with GDB}

If \program{OpenOCD} is running, a running program can be debugged using:

\begin{minted}{bash}
$ make debug
\end{minted}

This starts \program{GDB} and attaches to the SAM4S. \program{GDB} is
a command line debugger but there are many GUI programs that will
control it, for example, \program{vscode} and \program{geany} have
plugins.

\program{GDB} allows you to inspect the CPU registers, memory, set
breakpoints, set watchpoints, and much more.

You can reset your program using the GDB \code{jump reset} command.
However, this does not reset the peripherals as with a power-on reset.

The backtrace, \code{bt}, command is useful to show the function call
stack.

A handy overview of GDB's CLI commands can be found at:
\url{https://darkdust.net/files/GDB%20Cheat%20Sheet.pdf}.

\section{Interactive debugging with Visual Studio Code}

There is a VS Code debugger configuration provided in the wacky-racers
Git repository. It requires that you have the recommended extensions
installed first: open the command palette (Ctrl-Shift-P), type ``show
recommended extensions'', and install the extensions that are listed
in the sidebar. When \program{OpenOCD} is running (either via the
command line or the VS Code Ctrl-Shift-B build menu), press \verb|F5|
to start the debugger.

See the VS Code docs for more information on using the interactive
debugger: \url{https://code.visualstudio.com/docs/editor/debugging}.
